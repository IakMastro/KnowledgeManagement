\label{Chapter2}

\section{Θεωρητικό υπόβαθρο της διαχείρισης γνώσης}

\subsection{Διαχωρισμός δεδομένων, πληροφορίας και γνώσης}

\begin{problem}
  ``Δεδομένα: ουσιαστικοποιημένο ουδέτερο πληθυντικού της μετοχής δεδομένος, έχει δύο σημασίες:

  \begin{enumerate}
    \item στοιχεία, πληροφορίες, σε δυαδική μορφή που εισάγονται προς επεξεργασία σε έναν ηλεκτρονικό υπολογιστή ή προβάλλονται ως έξοδος σε μια περιφερειακή συσκευή
    \item στοιχεία, πληροφορίες, που έχουν λάβει δυαδική μορφή και έχουν αποθηκευτεί σε σκληρό δίσκο ή άλλο μέσο.
  \end{enumerate}

  Στην πληροφορική ότι αφορά το λογισμικό (όχι το υλισμικό) είναι δεδομένα. Ακόμη και τα προγράμματα που εκτελούνται εμπεριέχονται σε αρχεία, τα εκτελέσιμα αρχεία, που τα δεδομένα τους είναι εντολές για το τι πρέπει να κάνει το πρόγραμμα. Όταν ξεκινάει ο υπολογιστής φορτώνει από προκαθορισμένη θέση του σκληρού δίσκου τα δεδομένα που του λένε που θα βρει αποθηκευμένα τα εκτελέσιμα αρχεία του λειτουργικού συστήματος. Μετά το υλισμικό ότι υπάρχει είναι δεδομένα.''

  \href{https://el.wiktionary.org/wiki/%CE%B4%CE%B5%CE%B4%CE%BF%CE%BC%CE%AD%CE%BD%CE%B1}{\textbf{Ορισμός <<Δεδομένα>> στο Βικιλεξικό}}
\end{problem}

Τα δεδομένα τα οποία αρχικά είχαν συλλεχθεί ήταν τα exchange rates όλων των νομισμάτων σε άλλα νομίσματα.

\begin{problem}
  ``Πληροφορία: θηλυκό ουσιαστικό που στην θεωρία της πληροφορίας είναι η τυχαία τιμή ή ο συνδυασμός τιμών εντός της επιτρεπόμενης πληροφοριακής εντροπίας χωρίς αναγκαστικά να μεταφέρει σημασιολογικό κώδικα.''

  \href{https://el.wiktionary.org/wiki/%CF%80%CE%BB%CE%B7%CF%81%CE%BF%CF%86%CE%BF%CF%81%CE%AF%CE%B1}{\textbf{Ορισμός <<Πληροφορία>> στο Βικιλεξικό}}
\end{problem}

Πληροφορία είναι μία συλλογή δεδομένων που έχει επεξεργαστεί για να μπορεί να γίνει περισσότερη χρήση του και να βγει ένα αποτέλεσμα. Στο συγκεκριμένο παράδειγμα, κρατήθηκαν μόνο τα δεδομένα του Ευρώ και έτσι το νέο αρχείο ήταν συγκεντρωμένο με πιο καθαρές πληροφορίες σχετικές με το θέμα.

\begin{problem}
  ``Γνώση: θηλυκό ουσιαστικό που είναι οι πληροφορίες που αποκτά κάποιος και οι παραστάσεις που σχηματίζει για τον κόσμο και τα πράγματα μετά από την νοητική επεξεργασία των εμπειρικών δεδομένων''

  \href{https://el.wiktionary.org/wiki/%CE%B3%CE%BD%CF%8E%CF%83%CE%B7}{\textbf{Ορισμός <<Γνώση>> στο Βικιλεξικό}}
\end{problem}

Οι γνώσεις που μπορούν να θεωρηθούν από τα αποτελέσματα της εφαρμογής που μελετήθηκαν στο Κεφάλαιο~\ref{chap:results}. Με αυτά, μπορεί ο οποιοσδήποτε να παρατηρήσει ότι η διαφορά μεταξύ υψηλότερης και κατώτερης τιμής στα exchange offices παραμένουν παρόμοια κατά μέσο όρο στην πάροδο του χρόνου και μόνο σε συγκεκριμένες περιπτώσεις το θέμα γίνετε τραγικά κακό.

\subsection{Κύκλος ζωής διαχείρισης γνώσης}

Ο κύκλος ζωής της διαχείρισης γνώση αποτελείτε από τέσσερα στάδια.

\begin{itemize}
  \item Αποτύπωση της γνώσης
  \item Κωδικοποίηση της γνώσης
  \item Έλεγχος και Ανάπτυξη της γνώσης
  \item Διάχυση και Μεταφορά της γνώσης
\end{itemize}

\subsubsection{Αποτύπωση της γνώσης}

Αποτύπωση της γνώσης είναι το στάδιο στην διαχείριση της γνώσης όπου συγκεντρώνονται οι πληροφορίες και με την χρήση κατάλληλων εργαλείων μπορούν να αποτυπωθούν σε άλλες μορφές, όπως προγράμματα, βιβλία, άρθρα και άλλα.

\subsubsection{Κωδικοποίηση της γνώσης}

Κωδικοποίηση της γνώσης είναι το στάδιο στο οποίο μετατρέπεται η γνώση σε μορφή προσβάσιμη, ορατή και χρησιμοποιήσιμη για λήψη αποφάσεων. Για να μπορεί να υποστηριχτεί εξαγωγή συμπερασμάτων από τον υπολογιστή, θα πρέπει να γίνει σε μία γλώσσα που θα μπορεί να διαβαστεί και να καταλαβαίνει ο άνθρωπος. Με άλλα λόγια, το στάδιο της κωδικοποίησης της γνώσης χρησιμοποιείται για τους εξής λόγους χρήσης περιπτώσεων (use cases):

\begin{itemize}
  \item \textbf{Την ανάκτηση και αναζήτηση} της σχετικής γνώσης (είτε με σχέση ρητής και άρρητης, είτε με ένα μοντέλο matchmaking μεταξύ χρήστη και σχετικής γνώσης)
  \item \textbf{Την διάγνωση και αναγνώριση} των προσδιορίσιμων συμπτωμάτων
  \item \textbf{Την εκπαίδευση} νέων υπαλλήλων
  \item \textbf{Την εξαγωγή συμπερασμάτων} σχετικών με το πιθανό αποτέλεσμα
  \item \textbf{Λήψη αποφάσεων}
\end{itemize}

Βέβαια, θα πρέπει και ο ίδιος ο άνθρωπος να μπορεί να έχει τις κατάλληλες ικανότητες για να μπορεί να κατανοήσει την γνώση που του παρέχει το σύστημα σε αυτό το στάδιο. Χωρίς γνώσεις για την μέθοδο που ακολουθεί το σύστημα ή της θεματικής περιοχής και δίχως την ικανότητα λογικής σκέψης, αντίληψης και ανάπτυξη πρωτοτύπων, ο χρήστης δεν θα μπορεί να κατανοήσει το αποτέλεσμα. Υπάρχουν αρκετοί τρόποι ανάπτυξης του σταδίου, αλλά για το πλαίσιο της εργασίας θα μελετηθούν δύο: α) \textbf{ο τρόπος των <<κανόνων>>} και β) \textbf{ο τρόπος <<Case-Based Reasoning (CBR)>>}.

\subsubsection*{Κανόνες}

Η μεθοδολογία των κανόνων είναι μία πολύ πρακτική μεθοδολογία για την εξαγωγή των συμπερασμάτων και αποτελούν τη βάση πολλών ευφυών συστημάτων στον κόσμο. Τα πλεονεκτήματά του είναι τα εξής:

\begin{itemize}
  \item Είναι πιο κοντά στην ανθρώπινη γνώση, οπότε χρειάζεται λιγότερη προσπάθεια από τον χρήστη για να την κατανοήσει
  \item Υπάρχει επάρκεια συνεπαγωγών
  \item Κάθε κανόνας ορίζει ένα μικρό ανεξάρτητο τμήμα της γνώσης
  \item Μπορεί να προστεθεί νέος κανόνας οποιαδήποτε στιγμή
  \item Μπορεί να τροποποιηθεί οποιοσδήποτε κανόνας οποιαδήποτε στιγμή
\end{itemize}

Υπάρχουν τρεις μορφές κανόνων. Οι τρεις μορφές κανόνων παρουσιάζονται στον Πίνακα~\ref{tab:rules}

\begin{table}[H]
  \centering
  \begin{tabular}{| p{3cm} | p{3cm} | p{4cm} | p{5cm} |}
    \hline
    \textbf{Μορφές κανόνων} & \textbf{Εκφράζει} & \textbf{Επεξήγηση} & \textbf{Παράδειγμα} \\
    \hline
    IF συνθήκες THEN συμπέρασμα & Δηλωτική γνώση & ΑΝ οι συνθήκες αληθεύουν τότε αληθεύει και το συμπέρασμα & IF high > 2 * low THEN η διαφορά μεταξύ της υψηλότερης τιμής και της χαμηλότερης είναι ίση ή μεγαλύτερη της διπλάσιας \\
    \hline
    IF συνθήκες THEN ενέργειες & Διαδικαστική γνώση & ΑΝ οι συνθήκες αληθεύουν τότε να πραγματοποιηθούν ενέργειες & IF high > 2 * low THEN ερεύνα πίσω από το τι οφείλετε τόσο μεγάλη διαφορά μεταξύ του υψηλότερου και του χαμηλότερου exchange rate \\
    \hline
    ON συμβάν IF συνθήκες THEN ενέργειες & Διαδικαστική γνώση & Όταν συμβεί το γεγονός ΑΝ οι συνθήκες αληθεύουν τότε να πραγματοποιηθούν ενέργειες & ON high > 2 * low IF παρατηρηθεί εκμετάλλευση THEN παρουσίαση προβλήματος στον υπόλοιπο κόσμο \\
    \hline
  \end{tabular}
  \caption{Οι τρεις μορφές κανόνων}
  \label{tab:rules}
\end{table}

\subsubsection*{Case-Based Reasoning}

Η μέθοδος CBR αποτελεί μέθοδο συλλογιστικής που βασίζεται σε σχετικές παλαιές περιπτώσεις και προσομοιάζει την ανθρώπινη συλλογιστική βάσει εμπειριών. Στόχος της μεθόδου είναι να γίνει ανάδειξη των πιο σχετικών ιστορικών περιπτώσεων που ταιριάζουν με τη παρούσα περίπτωση. Για την περίπτωση που μελετήθηκε για την εργασία, θα μπορούσαν να μιλήσουν άτομα τα οποία τα εκμεταλλεύτηκαν στο παρελθόν παρόμοια γραφεία και να παρουσιάσουν το χρηματικό πόσο που δώσανε και αυτό που δέχτηκαν.

\subsubsection{Έλεγχος και Ανάπτυξη της γνώσης}

Στο στάδιο αυτό, γίνετε αναθεώρηση της υπάρχουσας εργασίας. Γενικώς, γίνονται διάφορα τεστ σε ρεαλιστικά περιβάλλοντα και σε πραγματικό κόσμο. Εξετάζονται τα εξής χαρακτηριστικά:

\begin{itemize}
  \item Υποκειμενικότητα γνώσης
  \item Δυσκολία στον έλεγχο αποτελεσματικής αποτύπωσής της
  \item Απουσία αναλυτικών λειτουργικών και τεχνικών προδιαγραφών
  \item Δυσκολία εδραίωσης συνέπειας και ορθότητας διαδικασιών ελέγχων
  \item Πολυπλοκότητα
\end{itemize}

Για τα συγκεκριμένα αποτελέσματα, θα γίνουν έλεγχοι για άχρηστες πληροφορίες και για λάθη πλεονασμού.

\subsubsection*{Άχρηστες πληροφορίες}

Άχρηστη πληροφορία ονομάζεται όταν μία συνθήκη για την επαλήθευση του κανόνα δεν επαληθεύεται ποτέ ή όταν οι συνθήκες είναι αντικρουόμενες. Για παράδειγμα, εάν το high είναι ίσο με το low, τότε για τα αποτελέσματα και τον σκοπό της ερεύνας δεν είναι κάπου χρήσιμα. Βέβαια, αυτό σημαίνει ότι όλα τα γραφεία έχουν συμφωνημένες τιμές μεταξύ τους και αυτό σημαίνει ότι δεν υπάρχει πρόβλημα για να λυθεί.

\subsubsection*{Λάθη πλεονασμού}

Τα λάθη πλεονασμού δημιουργούνται όταν υπάρχουν παραπάνω από μία μέθοδοι επίλυσης σε ένα πρόβλημα και δεν είναι ξεκάθαρο ποια λύση είναι η βέλτιστη. Στο συγκεκριμένο παράδειγμα, παρόλο που ακολουθήθηκε clustering, θα μπορούσε να έβγαζε πολύ καλύτερα αποτελέσματα ένας αντίστοιχος categorization αλγόριθμος σαν τον K-Nearest Neighbors εάν υπήρχαν κατηγοριοποιήσεις.

\subsubsection{Διάχυση και Μεταφορά της γνώσης}

Διάδοση και μεταφορά της γνώσης είναι το τελευταίο στάδιο της διαχείρισης της γνώσης. Τα προαπαιτούμενα του κεφαλαίου αυτού είναι:

\begin{itemize}
  \item Αναγνώριση της προσωπικής φύσης της γνώσης
  \item Δεν ισχύει: <<εφόσον αναπτυχθεί, θα χρησιμοποιηθεί>>
  \item Για μετάδοση γνώσης απαιτείται αλλαγή σε κουλτούρα και στάση
  \item Απαιτείται κλίμα εμπιστοσύνης
  \item Διαχείριση αλλαγής
  \item Κλίμα συνεργασίας και όχι ανταγωνισμού
  \item Άποψη διευθυντικού δυναμικού για διάχυση της γνώσης
  \item Αποτίμηση ικανοποίησης υπαλλήλων από εργασία και σταθερότητα.
\end{itemize}

Υπάρχουν δύο βασικές προσεγγίσεις για την μετάδοση της γνώσης που μπορούν να χρησιμοποιηθούν, α) η λειτουργο-κεντρική προσέγγιση και β) η περιεχομενο-κεντρική προσέγγιση.

\subsubsection*{Η λειτουργο-κεντρική προσέγγιση}

Η λειτουργο-κεντρική προσέγγιση είναι η προσέγγιση η οποία αντιλαμβάνεται τη μετάδοση της γνώσης κυρίως ως μία εκδήλωση κοινωνικής συναναστροφής και επικοινωνίας. Σε αυτή την προσέγγιση, η γνώση είναι στενά συνδεδεμένη με το άτομο που την κατέχει ενώ διαδίδεται κυρίως διαμέσου ανθρώπινων επαφών και σχέσεων. Ο κύριος λόγος των τεχνολογιών πληροφορικής έγκειται στην υποστήριξη της γνώσης μεταξύ ανθρώπων και όχι της αποθήκευσή της. Στην πράξη, θα μπορούσε να γίνει μία εκδήλωση κατά των εκμεταλλευτών όπου θα βγουν να κάνουν ομιλίες άτομα τα οποία υπήρξαν θύματα μιας τέτοιας απάτης και να πουν την εμπειρία τους σε κάποιο πλήθος.

\subsubsection*{Η περιεχομενο-κεντρική προσέγγιση}

Η περιεχομενο-κεντρική προσέγγιση εστιάζεται σε κωδικοποιημένη γνώση και τη μετάδοση της μέσα από πληροφοριακά συστήματα <<εταιρικής μνήμης>>. Για αυτό τον λόγο, θα μπορούσε να δημιουργηθεί μία δικτυακή πλατφόρμα που θα είναι ανεβασμένα έγγραφα και ομιλίες σχετικά με τα γραφεία που παρατηρείτε συχνή εκμετάλλευση.

\subsection{Μοντέλο αλλαγών του Kotter}

Ο σκοπός κάθε οργανισμού είναι να πουλήσει μία ιδέα και να την κάνει πράξη. Για να το πραγματοποιήσει αυτό, όμως, θα πρέπει να ακολουθήσει ένα μοντέλο. Ένα από αυτά τα μοντέλα είναι του Kotter, το οποίο μελετήθηκε στο μάθημα. Το \textbf{μοντέλο Kotter} περιγράφει το πως μπορεί ο καθένας με 8 βήματα να αναπτύξει το όραμά του και να το κάνει πραγματικότητα. Τα 8 βήματα είναι τα εξής:

\begin{itemize}
  \item Δημιουργία αίσθησης της αναγκαιότητας
  \item Δημιουργία του καθοδηγητικού συνασπισμού
  \item Ανάπτυξη οράματος και στρατηγικής
  \item Μετάδοση του οράματος για αλλαγή
  \item Εκχώρηση αρμοδιοτήτων στους εργαζόμενους για δράση σε πολλαπλά επίπεδα
  \item Δημιουργία βραχυπρόθεσμων επιτευγμάτων
  \item Παγίωση των ωφελειών και παραγωγή ακόμα περισσότερων αλλαγών
  \item Ενσωμάτωση νέων μεθόδων στην φιλοσοφία του οργανισμού
\end{itemize}

Θα ακολουθούν τα βήματα του μοντέλου Kotter για να παρουσιαστούν λύσεις για το πρόβλημα της εκμετάλλευσης.

\subsubsection{Βήμα 1: Δημιουργία αίσθησης της αναγκαιότητας}

\begin{problem}
  Σε αυτό το βήμα θα πρέπει να αναλυθεί ένα πρόβλημα το οποίο υπάρχει είτε εντός του οργανισμού και να γίνει μελέτη για το πως μπορεί να λυθεί αυτό το πρόβλημα.
\end{problem}

Το πρόβλημα έχει αναλυθεί στο Κεφάλαιο~\ref{chap:problem}, αλλά μία σύντομη περιγραφή του προβλήματος είναι για τους τουρίστες που αναγκάζονται να αλλάξουν τα χρήματά τους από το ένα νόμισμα στο άλλο και τους κινδύνους που αυτό εγκυμονεί.

\subsubsection{Βήμα 2: Δημιουργία του καθοδηγητικού συνασπισμού}

\begin{problem}
  Σε αυτό το βήμα θα πρέπει να πειστούν αρκετά πολλά άτομα, μερικά και σε υψηλόβαθμες θέσεις στην κοινωνία, έτσι ώστε να δημιουργηθεί μία φωνή για να ξεκινήσει η αλλαγή. Πρέπει να δημιουργηθεί ένα είδος επείγουσας αλλαγής και μία ανάγκη προς αυτήν.
\end{problem}

Η ομάδα μπορεί να δημιουργηθεί αρκετά εύκολα στην εποχή της πληροφορίας και του διαδικτύου. Μπορούν να δημιουργηθούν σε διάφορα δίκτυα διασυνδέσεις στο διαδίκτυο, όπως το reddit και το twitter, να παρουσιαστεί το πρόβλημα, να αναφερθούν παραδείγματα που έχουν βιντεοσκοπηθεί και ανεβεί στο YouTube ή καταγραφεί στο Reddit και να δημιουργηθεί μία συζήτηση στην οποία θα οδηγήσει στην δημιουργία κάποιας ομάδας.

\subsubsection{Βήμα 3: Ανάπτυξη οράματος και στρατηγικής}

\begin{problem}
  Σε αυτό το βήμα θα πρέπει να ξεκαθαριστεί ποια είναι η ιδέα και η λύση του προβλήματος που προσπαθεί η ομάδα να λύσει. Στο τέλος του βήματος, θα πρέπει να μπορεί να παρουσιαστεί η τελική ιδέα μέσα σε πέντε λεπτά ή λιγότερο μέσω μίας ομιλίας.
\end{problem}

Αυτό είναι το πιο δύσκολο μέρος του προβλήματος. Όλα κρίνονται από το πως θα παρουσιαστεί στο ευρύ κοινό. Το πιο σημαντικό είναι να γίνει επίκληση στο συναίσθημα και όχι στην λογική. Οι άνθρωποι τείνουν στο να είναι συμπονετικοί προς τις δυσκολίες που περνάει ένας συνάνθρωπός τους, κυρίως σε καταστάσεις που θα μπορούσε να περάσει και ο ίδιος, όπως στο παράδειγμα των γραφείων ανταλλαγών που επιλέγουν να κλέβουν λεφτά από τους πελάτες. Μπορεί να προταθεί να οριστεί ένα συγκεκριμένο μέγιστο ποσοστό που θα μπορούν να χρεώνουν τα γραφεία και να δίνουν πρόστιμο σε όσους το παραβιάζουν.

\subsubsection{Βήμα 4: Μετάδοση του οράματος για αλλαγή}

\begin{problem}
  Σε αυτό το βήμα θα πρέπει να μεταφερθεί το όραμα όσο πιο συχνά γίνετε και με κάθε ευκαιρία.
\end{problem}

Καθώς η τρέχουσα εποχή του Διαδικτύου επιτρέπει την εύκολη επικοινωνία μεταξύ πολιτών μέσα από τα social media, δεν χρειάζεται να υπάρξει κάποια δημόσια ομιλία δια ζώσης εξ αρχής. Μέσω των social media, μπορούν να διοργανωθούν αρκετές ζωντανές αναμεταδόσεις με ανθρώπους να μιλάνε για τον προβληματισμό τους και να προτείνουν λύσεις για να μπορεί να επιταχυνθεί. Στην συνέχεια, μπορεί να διοργανωθεί μία παρουσίαση σε ένα TedTalk για να καλυφθεί και περισσότερος κόσμος και να ακούσει το πρόβλημα.

\subsubsection{Βήμα 5: Εκχώρηση αρμοδιοτήτων στους εργαζόμενους για δράση σε πολλαπλά επίπεδα}

\begin{problem}
  Σε αυτό το βήμα θα πρέπει να βρεθούν τα σωστά άτομα για να επιτελέσουν ηγετικό ρόλο στην ιδέα. Αυτό γίνετε βρίσκοντας νέα άτομα τα οποία έχουν γνώσεις πάνω στο θέμα που γίνετε προσπάθεια για την αλλαγή, τα οποία θα πρέπει να επιβραβευτούν για κάθε καλό που κάνουν. Θα πρέπει να είναι από όλες τις κοινωνικές τάξεις και να μην υπάρχουν διακρίσεις.
\end{problem}

Για το πρόβλημα της εκμετάλλευσης, το πρόβλημα μπορεί να λυθεί από την Ε.Ε. και από τους ίδιους τους πολίτες. Θα πρέπει να ενσωματωθούν στο όραμα διάφοροι πολιτικοί, με σκοπό να φτάσει το θέμα στην Βουλή. Οποίον καταφέρουν να πείσουν, θα βραβευτεί από τους ηγέτες ως δωρεάν διαφήμιση προς αυτόν και να γίνουν χορηγοί του. Ο πολιτικός θα είναι υπεύθυνος να μεταφέρει την ιδέα μέσα στην Βουλή και να αρχίζει να παρουσιάζει τις λύσεις τους οποίους προτείνει η ομάδα. Από την πλευρά των πολιτών, ευθύνη τους είναι να ενημερώνουν τους τουρίστες με σκοπό να τους αποφεύγουν. Αυτό θα τους χτυπήσει οικονομικά και θα αναγκαστούν να το σταματήσουν σε μία στιγμή.

\subsubsection{Βήμα 6: Δημιουργία βραχυπρόθεσμων επιτευγμάτων}

\begin{problem}
  Σε αυτό το βήμα θα πρέπει να φτάσει η ομάδα σε έναν από τους σκοπούς της. Η δημιουργία μικρών νικών ανεβάζει το ήθος των μελών της ομάδας και κάνει την αλλαγή πιο εύκολη να εφαρμοστεί για τον μέσο ενδιαφερόμενο.
\end{problem}

Ένα μεγάλο βραχυπρόθεσμο επίτευγμα είναι να σταματήσει ένα γραφείο ανταλλαγών να εκμεταλλεύεται τον κόσμο και να μειώσει την τιμή που του κρατάει.

\subsubsection{Βήμα 7: Παγίωση των ωφελειών και παραγωγή ακόμα περισσότερων αλλαγών}

\begin{problem}
  Σε αυτό το βήμα θα πρέπει να χτίσουν πάνω στις μικρές επιτυχίες του προηγούμενου βήματος αναγνωρίζοντας ότι οι νίκες δεν είναι αυτόνομες και χρειάζονται επεκτάσεις για να ολοκληρωθεί.
\end{problem}

Μπορούν να γίνουν προτάσεις σε παγκόσμιο πεδίο για να αλλάξει όλος ο κόσμος σε ένα κοινό νόμισμα, όπως έχει κάνει η Ευρώπη.

\subsubsection{Βήμα 8: Ενσωμάτωση νέων μεθόδων στην φιλοσοφία του οργανισμού}

\begin{problem}
  Σε αυτό το βήμα θα πρέπει να ενσωματωθεί το όραμα πλήρως στην κοινωνία, να παρατηρηθούν τα λάθη του και να βρεθούν νέα προβλήματα προς επίλυση. Θα πρέπει επίσης να υπάρχουν συχνές αναφορές στις νίκες του οργανισμού.
\end{problem}

Το τελευταίο βήμα είναι της συντήρησης και της παρατήρησης των νέων προβλημάτων που μπορεί να δημιουργηθούν. Συνήθως, οι εκμεταλλευτές βρίσκουν νέους τρόπους για να κλέψουν χρήματα από τους πελάτες τους χωρίς να φαίνεται κάπου και τελείως νόμιμα.