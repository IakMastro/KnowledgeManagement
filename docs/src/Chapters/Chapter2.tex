\label{Chapter2}

\section{Θεωρητικό υπόβαθρο της διαχείρισης γνώσης}

\subsection{Διαχωρισμός δεδομένων, πληροφορίας και γνώσης}

\begin{problem}
  ``Δεδομένα: ουσιαστικοποιημένο ουδέτερο πληθυντικού της μετοχής δεδομένος, έχει δύο σημασίες:

  \begin{enumerate}
    \item στοιχεία, πληροφορίες, σε δυαδική μορφή που εισάγονται προς επεξεργασία σε έναν ηλεκτρονικό υπολογιστή ή προβάλλονται ως έξοδος σε μια περιφερειακή συσκευή
    \item στοιχεία, πληροφορίες, που έχουν λάβει δυαδική μορφή και έχουν αποθηκευτεί σε σκληρό δίσκο ή άλλο μέσο.
  \end{enumerate}

  Στην πληροφορική ότι αφορά το λογισμικό (όχι το υλισμικό) είναι δεδομένα. Ακόμη και τα προγράμματα που εκτελούνται εμπεριέχονται σε αρχεία, τα εκτελέσιμα αρχεία, που τα δεδομένα τους είναι εντολές για το τι πρέπει να κάνει το πρόγραμμα. Όταν ξεκινάει ο υπολογιστής φορτώνει από προκαθορισμένη θέση του σκληρού δίσκου τα δεδομένα που του λένε που θα βρει αποθηκευμένα τα εκτελέσιμα αρχεία του λειτουργικού συστήματος. Μετά το υλισμικό ότι υπάρχει είναι δεδομένα.''

  \href{https://el.wiktionary.org/wiki/%CE%B4%CE%B5%CE%B4%CE%BF%CE%BC%CE%AD%CE%BD%CE%B1}{\textbf{Ορισμός <<Δεδομένα>> στο Βικιλεξικό}}
\end{problem}

Τα δεδομένα τα οποία αρχικά είχαν συλλεχθεί ήταν τα exchange rates όλων των νομισμάτων σε άλλα νομίσματα.

\begin{problem}
  ``Πληροφορία: θηλυκό ουσιαστικό που στην θεωρία της πληροφορίας είναι η τυχαία τιμή ή ο συνδυασμός τιμών εντός της επιτρεπόμενης πληροφοριακής εντροπίας χωρίς αναγκαστικά να μεταφέρει σημασιολογικό κώδικα.''

  \href{https://el.wiktionary.org/wiki/%CF%80%CE%BB%CE%B7%CF%81%CE%BF%CF%86%CE%BF%CF%81%CE%AF%CE%B1}{\textbf{Ορισμός <<Πληροφορία>> στο Βικιλεξικό}}
\end{problem}

Πληροφορία είναι μία συλλογή δεδομένων που έχει επεξεργαστεί για να μπορεί να γίνει περισσότερη χρήση του και να βγει ένα αποτέλεσμα. Στο συγκεκριμένο παράδειγμα, κρατήθηκαν μόνο τα δεδομένα του Ευρώ και έτσι το νέο αρχείο ήταν συγκεντρωμένο με πιο καθαρές πληροφορίες σχετικές με το θέμα.

\begin{problem}
  ``Γνώση: θηλυκό ουσιαστικό που είναι οι πληροφορίες που αποκτά κάποιος και οι παραστάσεις που σχηματίζει για τον κόσμο και τα πράγματα μετά από την νοητική επεξεργασία των εμπειρικών δεδομένων''

  \href{https://el.wiktionary.org/wiki/%CE%B3%CE%BD%CF%8E%CF%83%CE%B7}{\textbf{Ορισμός <<Γνώση>> στο Βικιλεξικό}}
\end{problem}

Οι γνώσεις που μπορούν να θεωρηθούν από τα αποτελέσματα της εφαρμογής που μελετήθηκαν στο Κεφάλαιο~\ref{chap:results}. Με αυτά, μπορεί ο οποιοσδήποτε να παρατηρήσει ότι η διαφορά μεταξύ υψηλότερης και κατώτερης τιμής στα exchange offices παραμένουν παρόμοια κατά μέσο όρο στην πάροδο του χρόνου και μόνο σε συγκεκριμένες περιπτώσεις το θέμα γίνετε τραγικά κακό.

\subsection{Κύκλος ζωής διαχείρισης γνώσης}

\subsection{Μοντέλο αλλαγών του Kotter}
