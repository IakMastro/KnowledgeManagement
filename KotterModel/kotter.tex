\documentclass{fphw}

\usepackage{ucs}

\usepackage[utf8x]{inputenc}
\usepackage[greek, english]{babel}
\usepackage{alphabeta}
\usepackage{lmodern}

\usepackage[T1]{fontenc}
\usepackage{graphicx}
\usepackage{booktabs}
\usepackage{listings}
\usepackage{enumerate}

\title{Προαιρετική εργασία - Το μοντέλο του Kotter}
\author{Ιάκωβος Μαστρογιαννόπουλος}
\institute{Πανεπιστημιο Δυτικης Αττικης \\ Τμημα Πληροφορικης Και Υπολογιστων}
\class{Διαχείριση Γνώσης}
\professor{Αθανάσιος Κιούρτης}

\begin{document}
    \maketitle

    \section*{Περιγραφή του μοντέλου Kotter}
    Ο σκοπός κάθε οργανισμού είναι να πουλήσει μία ιδέα και να την κάνει πράξη. Για να το πραγματοποιηθεί
    αυτό, όμως, θα πρέπει να ακολουθήσει ένα μοντέλο. Ένα από αυτά τα μοντέλα
    είναι του Kotter, το οποίο μελετήθηκε στο μάθημα.
    Το \textbf{μοντέλο Kotter} περιγράφει το πως μπορεί ο καθένας με 
    8 βήματα να αναπτύξει το όραμα του και να το κάνει πραγματικότητα.
    Τα 8 βήματα είναι τα εξής:
    \begin{itemize}
        \item Δημιουργία αίσθησης της αναγκαιότητας
        \item Δημιουργία του καθοδηγητικού συνασπισμού
        \item Ανάπτυξη οράματος και στρατηγικής
        \item Μετάδοση του οράματος για αλλαγή
        \item Εκχώρηση αρμοδιοτήτων στους εργαζόμενους για δράση σε πολλαπλά επίπεδα
        \item Δημιουργία βραχυπρόθεσμων επιτευγμάτων
        \item Παγίωση των ωφελειών και παραγωγή ακόμα περισσότερων αλλαγών
        \item Ενσωμάτωση νέων μεθόδων στην φιλοσοφία του οργανισμού
    \end{itemize}
    Θα ακολουθήσει παράδειγμα μέσω ενός προβλήματος της καθημερινής ζωής για το πως
    μπορεί να ακολουθηθεί το μοντέλο Kotter για την βελτίωση της ζωής.

    \section*{Βήμα 1: Δημιουργία αίσθησης της αναγκαιότητας}

    \begin{problem}
        Σε αυτό το βήμα θα πρέπει να αναλυθεί ένα πρόβλημα το οποίο υπάρχει
        είτε μέσα του οργανισμού και να γίνει μελέτη για το πως μπορεί να 
        λυθεί αυτό το πρόβλημα.
    \end{problem}

    \subsection*{Το πρόβλημα}
    Το πρόβλημα που θα αναλυθεί και θα λυθεί με το μοντέλο του Kotter είναι
    η γραφειοκρατεία σχετικά με όλους τους κωδικούς και την πλαστική ταυτότητα που έχει ένας Έλληνας πολίτης.
    Όποιος έχει ταξιδέψει στο εξωτερικό και συγκεκριμένα σε χώρες της Ευρωπαϊκής Ένωσης
    όπου δεν χρειάζεται να έχει εκδοθεί διαβατήριο για να μπορεί να περάσει στην χώρα,
    θα του έχει τύχει έστω και φορά, στον έλεγχο των στοιχείων του, να βλέπουν οι υπεύθυνοι
    την ταυτότητα και να γελάνε με το πόσο απαρχαιωμένη είναι. Αυτό συμβαίνει επειδή σε αρκετές 
    χώρες του εξωτερικού έχουν αντικαταστήσει τις πλαστικές τους κάρτες με ηλεκτρικές κάρτες, 
    όπου έχουν πάνω τους ένα chip που περιέχει όλους τους κωδικούς που μπορούν να χρειαστούν στην
    καθημερινή τους ζωή. Έτσι, λχ ο Πολωνός που στην χώρα του έχει μία ηλεκτρική κάρτα γλυτώνει πολύ χρόνο
    όταν χρειαστεί να κάνει κάτι σχετικό με το Δημόσιο. Πόσες φόρες έχουν ακουστεί παράπονα από έλληνα πολίτη που τρέχει
    σε διάφορες δημόσιες υπηρεσίες για να βγάλει ένα χαρτί που θα μπορούσε να το έβγαζε πολύ εύκολα εάν
    δεν υπήρχε τόσο γραφειοκρατεία; Η απάντηση είναι πολλές φόρες. Βέβαια, μέσω της πανδημίας του Covid19, μερικά
    προβλήματα έχουν ήδη λυθεί, αρκετές υπηρεσίες έχουν γίνει ήδη ηλεκτρονικές και άλλες πρόκειται να ακολουθήσουν.
    Παρόλα αυτά, συνεχίζει να υπάρχει το πρόβλημα με τους πολλούς μοναδικούς αριθμούς που έχει ο κάθε πολίτης.

    \newpage
    \section*{Βήμα 2: Δημιουργία του καθοδηγητικού συνασπισμού}

    \begin{problem}
        Σε αυτό το βήμα θα πρέπει να πειστούν αρκετά πολλά άτομα, μερικά και σε υψηλόβαθμες θέσεις στην κοινωνία, 
        έτσι ώστε να δημιουργηθεί μία φωνή για να ξεκινήσει η αλλαγή. Πρέπει να δημιουργηθεί ένα είδος επείγουσας αλλαγής
        και μία ανάγκη προς αυτήν.
    \end{problem}

    \subsection*{Πως μπορεί να δημιουργηθεί μία τέτοια ομάδα}
    Η δημιουργία μίας ηγετικής ομάδας στις μέρες μας είναι αρκετά εύκολη. Μπορούν να δημιουργηθούν διάφορες 
    δημοσιεύσεις σε διάφορα δίκτυα διασυνδέσεις στο διαδίκτυο, όπως το reddit και το twitter.
    Για αρχή, θα πρέπει να παρουσιαστεί το πρόβλημα. Μερικοί που έχουν ήδη φτάσει σε ένα σημείο αγανάκτησης και κούρασης
    με το συγκεκριμένο θέμα της γραφειοκρατίας, θα διαβάσουν τα forums και θα ζητήσουν να συμμετάσχουν και αυτοί στην ομάδα. 
    Έτσι, θα μπορέσει να δημιουργηθεί μία αρχική ομάδα από κάποια συγκεκριμένα άτομα, 
    τα οποία θα αποτελέσουν ηγέτες της αλλαγής και θα προσπαθήσουν να απευθυνθούν στο συναίσθημα του πολίτη.

    \section*{Βήμα 3: Ανάπτυξη οράματος και στρατηγικής}

    \begin{problem}
        Σε αυτό το βήμα θα πρέπει να υπάρξει να ξεκαθαριστεί πιο είναι η ιδέα και η λύση του προβλήματος
        που προσπαθεί η ομάδα να λύσει. Στο τέλος του βήματος, θα πρέπει να μπορεί να παρουσιαστεί η τελική ιδέα μέσα σε πέντε λεπτά
        ή λιγότερο μέσω μίας ομιλίας.
    \end{problem}

    \subsection*{Λύση του προβλήματος}
    Για να λυθεί το θέμα με την γραφειοκρατία, τους πολλαπλούς μοναδικούς κωδικούς και της πλαστικής κάρτας, 
    θα πρέπει να υπάρξουν παραδείγματα για το πως είναι η ζωή ενός έλληνα πολίτη και να συγκριθεί με την ζωή ενός
    ευρωπαίου πολίτη. Θα πρέπει στο τέλος της ομιλίας να χτυπήσει στα συναισθήματα του ακροατή και να παρουσιαστεί μία λύση.
    Αυτή λύση δεν είναι άλλη, παρά να ζητηθεί να έχει ο έλληνας πολίτης έναν μοναδικό του αριθμό μητρώου, ο οποίος υπάρχει
    μέσα σε μία ηλεκτρονική ταυτότητα, μπορεί να κάνει τα πάντα με αυτόν και δεν χρειάζεται να τον ξέρει απέξω.
    
    \section*{Βήμα 4: Μετάδοση του οράματος για αλλαγή}

    \begin{problem}
        Σε αυτό το βήμα θα πρέπει να μεταφερθεί το όραμα όσο πιο συχνά γίνετε, με κάθε ευκαιρία.
    \end{problem}

    \subsection*{Η συχνή συζήτηση του οράματος είναι το κλειδί του βήματος}
    Δεν χρειάζεται να υπάρχει πάντα κάποια επίσημη ομιλία σχετικά με το όραμα. Όποτε βρεθεί έστω και ένα μικρό παραθυράκι
    για να μπορεί να γίνει συζήτηση πάνω του θέματος, είναι η κατάλληλη ώρα. Μπορεί να είναι παντού, δηλαδή στο σπίτι, στην δουλεία,
    στα μεταφορικά μέσα. Επιπρόσθετα, στην εποχή του διαδικτύου μπορούν να δημιουργηθούν και διαφημίσεις που θα παίζουν πάνω στο θέμα.
    Να φτιαχτεί ένα βίντεο για το YouTube, για παράδειγμα, το οποίο θα γίνετε παρωδία της σημερινής κατάστασης της γραφειοκρατίας και να
    παρουσιάζει τις λύσεις στο τέλος. Αυτό θα ανοίξει περισσότερες συζητήσεις πάνω στο θέμα και θα φτάσει σε ακόμα πιο πολλά άτομα.

    \section*{Βήμα 5: Εκχώρηση αρμοδιοτήτων στους εργαζόμενους για δράση σε πολλαπλά επίπεδα}

    \begin{problem}
        Σε αυτό το βήμα θα πρέπει να βρεθούν τα σωστά άτομα για να επιτελέσουν ρόλο
        ηγετικό στην ιδέα. Αυτό γίνετε βρίσκοντας νέα άτομα τα οποία έχουν γνώσεις πάνω στο θέμα
        που γίνετε προσπάθεια για την αλλαγή, τα οποία θα πρέπει να επιβραβευτούν για κάθε καλό που κάνουν.
        Θα πρέπει να είναι από όλες τις κοινωνικές τάξεις και να μην υπάρχουν διακρίσεις.
    \end{problem}

    \subsection*{Η εύρεση για τους σωστούς ανθρώπους}
    Για το πρόβλημα της γραφειοκρατίας, το πρόβλημα μπορεί να λυθεί μόνο από την ελληνική κυβέρνηση.
    Θα πρέπει να ενσωματωθούν στο όραμα διάφοροι πολιτικοί, με σκοπό να φτάσει το θέμα στην βουλή.
    Οποίον καταφέρουν να πείσουν, θα βραβευτεί από τους ηγέτες ως δωρεάν διαφήμιση προς αυτόν και να γίνουν χορηγοί του.
    Ο πολιτικός θα είναι υπεύθυνος να μεταφέρει την ιδέα μέσα στην βουλή και να αρχίζει να παρουσιάζει τις λύσεις
    τους οποίους προτείνει η ομάδα.

    \newpage
    \section*{Βήμα 6: Δημιουργία βραχυπρόθεσμων επιτευγμάτων}

    \begin{problem}
        Σε αυτό το βήμα θα πρέπει να φτάσει η ομάδα σε έναν από τους σκοπούς της.
        Η δημιουργία μικρών νίκων ανεβάζει το ήθος των μελών της ομάδας και κάνει την αλλαγή
        πιο εύκολη να εφαρμοστεί για τον μέσο ενδιαφερόμενο.
    \end{problem}

    \subsection*{Η πρώτη μικρή λύση του προβλήματος}
    Μία πρώτη λύση του προβλήματος είναι να πραγματοποιηθεί το βήμα της αλλαγής των μοναδικών αριθμών που
    υπάρχει τώρα σε έναν. Θα δημιουργηθούν ελάχιστα προβλήματα αντιστοίχησης των πολιτών με τους νέους αριθμούς μητρώων
    για το οποίο θα υπάρχει ελάχιστη χρέωση προς το ελληνικό δημόσιο για την αλλαγή των κωδικών.
    Όταν θα τελειώσει η συγκεκριμένη διαδικασία αλλαγής των κωδικών, θα υπάρξει μία μικρή νίκη προς τους πολίτες και τον οργανισμό,
    εφόσον θα υπάρχει μία μεγάλη θετική αντίδραση και θα ανέβει το ήθος της ομάδας και θα μπορέσει να συνεχίσει τον σκοπό της.

    \section*{Βήμα 7: Παγίωση των ωφελειών και παραγωγή ακόμα περισσότερων αλλαγών}

    \begin{problem}
        Σε αυτό το βήμα θα πρέπει να χτίσουν πάνω στις μικρές επιτυχίες του προηγούμενου βήματος
        αναγνωρίζοντας ότι οι νίκες δεν είναι αυτόνομες και χρειάζονται επεκτάσεις για να ολοκληρωθεί.
    \end{problem}

    \subsection*{Η μεγάλη νίκη του προβλήματος}
    Εφόσον η μικρή αλλαγή από τους πολλούς κωδικούς στον έναν, θα πρέπει να ξεκινήσει η διαδικασία αλλαγής
    της ταυτότητας από την τωρινή πλαστική σε νέες ταυτότητες, ηλεκτρονικές με chip στο οποίο βρίσκονται όλες οι πληροφορίες
    του πολίτη. Εννοείται πως χρειάζεται να υπάρχει μία μεταβατική περίοδος όπου και οι δύο ταυτότητες είναι έγκυρες, όπως 
    είχε γίνει και στην αλλαγή από Δραχμές σε Ευρώ. Θα πρέπει να γίνει σωστή ενημέρωση προς τους πολίτες και να τους πεισουν 
    ότι παρότι θα ταλαιπωρηθούν λίγο στην αλλαγή της ταυτότητας, περιμένοντας στην ουρά, μακροπρόθεσμα θα έχουν παραπάνω πλεονεκτήματα 
    από ότι έχουν με το τωρινό μοντέλο.

    \section*{Βήμα 8: Ενσωμάτωση νέων μεθόδων στην φιλοσοφία του οργανισμού}

    \begin{problem}
        Σε αυτό το βήμα θα πρέπει να ενσωματωθεί το όραμα πλήρως στην κοινωνία,
        να παρατηρηθούν τα λάθη του και να βρεθούν νέα προβλήματα προς επίλυση. Θα πρέπει
        επίσης να υπάρχουν συχνές αναφορές στις νίκες του οργανισμού.
    \end{problem}

    \subsection*{Πλήρη εφαρμογή του οράματος}
    Ο σκοπός της ομάδας είναι πετυχημένος. Θα πρέπει όμως να παρατηρηθούν οι προβληματισμοί των πολιτών
    όπως και προβλήματα που σχετίζονται με τις νέες ταυτότητες. Επίσης, καλό θα ήταν να παρατηρηθούν περισσότερα προβλήματα της κοινωνίας
    τα οποία θα μπορούσαν να λυθούν και να υπάρχει και συχνή σύγκριση της ζώης πριν τις ηλεκτρικές ταυτότητες και μετά.

    \section*{Βιβλιογραφία}
    \begin{itemize}
        \item Διαφάνειες Διαχείρισης Γνώσης, Α. Κιούρτης, 2021
    \end{itemize}
\end{document}